\documentclass{ctexart}

\usepackage{graphicx}
\usepackage{amsmath}
\title{Homework 5 Explain}

\author{王麟 \\ 数学与应用数学 3210104213}
\begin{document}
\maketitle

\begin{verbatim}
using brent method
 iter [    lower,     upper]      root        err  err(est)
    1 [1.0000000, 5.0000000] 1.0000000 -1.2360680 4.0000000
    2 [1.0000000, 3.0000000] 3.0000000 +0.7639320 2.0000000
    3 [2.0000000, 3.0000000] 2.0000000 -0.2360680 1.0000000
    4 [2.2000000, 3.0000000] 2.2000000 -0.0360680 0.8000000
    5 [2.2000000, 2.2366300] 2.2366300 +0.0005621 0.0366300
Converged:
    6 [2.2360634, 2.2366300] 2.2360634 -0.0000046 0.0005666
\end{verbatim}

通过迭代计算逼近$\sqrt{5}$直到相对误差小于设定值0.001,其中root为离目标更近的数,err为root-$\sqrt{5}$,err(est)=upper-lower,下一次叠代根据算法从上一组数据中产生原值。\\
程序也可以改变初始数据来逼近任何开方数直到要求的精度

\end{document}
