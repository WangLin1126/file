\documentclass{ctexbeamer}        

\usefonttheme{serif}              
\usefonttheme{professionalfonts}  
\usepackage{graphicx}
\usepackage{listings}
\usepackage{mathtools}
\usepackage{hyperref}
\usepackage{url}
\usepackage{tikz}
\usepackage{amsmath}
\usepackage[style=gb7714-2015]{biblatex}
\usepackage{subfigure}
\usepackage{gnuplot-lua-tikz}
\usepackage{float}
\usepackage{xcolor}
\hypersetup{pdfpagemode=FullScreen}  
\usetikzlibrary{shapes,arrows}
\usetheme{Frankfurt}       
\usecolortheme{orchid}     

\begin{document}
\lstset{
 columns=fixed,       
 numbers=left,                                            
 commentstyle=\it\color[RGB]{96,6,96},              
 stringstyle=\rmfamily\slshape\color[RGB]{0,128,0},   
}

\title{Julia集的分析和探索}
\author{王麟 \\ 数学与应用数学\\3210104213}
\institute{浙江大学}
\date{2022~年~7~月~4~日}
\frame{\titlepage}



\begin{frame}{目录} 
\begin{center}        
  \tableofcontents[hideallsubsections]
  \end{center}
\end{frame}



\section{引言}    
\indent Julia集合是以法国数学家加斯顿·朱莉娅(Gaston Julia)的名字命名的,他在1915年研究了这些集合的性质,并在1918年发表了著名的论文《理性基金上的Mémoire sur l'itération des fonctions rationnelles》。虽然Julia集现在与二次多项式$z_{n+1}=z^2_n+c$相关联,但Julia对更一般表达式的迭代性质感兴趣,即
$$z^4+\frac{z^3}{z-1}+\frac{z^2}{z^3+4z^2+5}+c$$

Julia集可以有各种形状,CCA中的一个小变化可以极大地改变Julia集。1979年,在计算机的帮助下,B.B.Mandelbrot研究了Julia集,试图对所有可能的形状进行分类,并提出了一种新的形状:Mandelbrot集。\\
\indent 在过去我们讨论了Mandelbrot集递归式,这是二次递归方程$z_{n+1}=z^2_n+z_0$(c是一个固定的复数)的特例。如今我们尝试使用类似Mandelbrot集的递归式进一步分析探索更普遍化的Julia集。

\AtBeginSection[]{\frame{\tableofcontents[currentsection,hideallsubsections]}}

\section{算法}  
\begin{frame}
\begin{center}
设置迭代次数与收敛半径、次数和常数c,设置一个复数点集为初始点集 ,代入递推公式计算 ,找出不发散的点,记录这些点的复数域平面直角坐标阵,重复叠代步骤至到达递退上限或者发散,画出坐标 ,即Julia集的图像。
\end{center}
\end{frame}
\AtBeginSection[]{\frame{\tableofcontents[currentsection,hideallsubsections]}}

\section{流程图展示}    

\begin{frame}
\tikzstyle{startstop} = [rectangle,rounded corners, minimum width=3cm,minimum height=1cm,text centered, draw=black]
\tikzstyle{io} = [trapezium, trapezium left angle = 70,trapezium right angle=110,minimum width=3cm,minimum height=1cm,text centered,draw=black]
\tikzstyle{process} = [rectangle,minimum width=3cm,minimum height=1cm,text centered,text width =3cm,draw=black]
\tikzstyle{decision} = [diamond,minimum width=3cm,minimum height=1cm,text centered,draw=black]
\tikzstyle{arrow} = [thick,->,>=stealth]

\begin{tikzpicture}[node distance=1.75cm]
\node (start) [startstop] {开始};
\node (process0) [process,below of=start] {给定c的值};
\node (process1) [process,below of=process0] {$z_{n+1}=z_n+c$};
\node (decision1) [decision,below of=process1,yshift=-0.5cm] {$|z_{n+1}|>2?$};
\node (decision2) [decision,right of=decision1,xshift=2cm] {$n<N$};
\node (process3) [process,right of=process1,xshift=2cm] {$n=n+1$};
\node (out1) [io,below of=decision1,yshift=-0.5cm] {发散};
\node (out2) [io,right of= out1,xshift=2cm] {不发散};
\node (stop) [startstop,below of=out1] {结束};

\draw [arrow] (start) -- (process0);
\draw [arrow] (process0) -- (process1);
\draw [arrow] (process1) -- (decision1);
\draw [arrow] (decision1) -- node[anchor=west] {Yes}(out1);
\draw [arrow] (decision1) -- node[anchor=south] {No} (decision2);
\draw [arrow] (decision2) -- node[anchor=east] {No} (out2);
\draw [arrow] (decision2) -- node[anchor=east] {Yes} (process3);
\draw [arrow] (process3) -- (process1);
\draw [arrow] (out1) -- (stop);
\draw [arrow] (out2) |- (stop);
\end{tikzpicture}

\end{frame}

\section{Julia set 效果图展示}
   \begin{figure}[htbp]
\centering

\subfigure{
\begin{minipage}[t]{0.3\linewidth}
\centering
\includegraphics[width=1.4in]{./photo/-0.80.1560.bmp}

\end{minipage}
}
\subfigure{
\begin{minipage}[t]{0.3\linewidth}
\centering
\includegraphics[width=1.4in]{./photo/-0.70.16.bmp}

\end{minipage}
}

\subfigure{
\begin{minipage}[t]{0.3\linewidth}
\centering
\includegraphics[width=1.4in]{./photo/-0.50.5.bmp}

\end{minipage}
}
\subfigure{
\begin{minipage}[t]{0.3\linewidth}
\centering
\includegraphics[width=1.4in]{./photo/0-0.5.bmp}

\end{minipage}
}
\end{figure}
\centering
\indent 以上四幅图分别通过改变dimension改变图片缩放大小 与c的初值改变图片形状而生成,而将叠代上限N不断扩大后生成的图片会愈发精细,局部细节也将产生重复。\\
\begin{frame}
\begin{figure}[htbp]
\centering
\subfigure{
\begin{minipage}[t]{0.3\linewidth}
\centering
\includegraphics[width=1.4in]{./photo/0.255.bmp}
\end{minipage}
}
\subfigure{
\begin{minipage}[t]{0.3\linewidth}
\centering
\includegraphics[width=1.4in]{./photo/-0.80.1.bmp}

\end{minipage}
}
\end{figure}
\indent 根据叠代次数分类将边界的外部或者内部填充色彩后,其展现出了渐变色彩,这说明原本属于同一类边界内部的图像也可以进一步分类,值得注意的是,我们尝试把所有的色彩填充至内部后,所有的位图颜色均能显现。这与Manderbrot展现出来的效果基本一致。
\end{frame}
\end{document}





